\chapter{Introduction} \label{chap::intro}

    Welcome to the standard layout for your IDEA LEAGUE MSc thesis written in \LaTeX. \LaTeX\  has a variety of advantages over conventional/ standard text editing programs, which you will soon enough discover yourself. \LaTeX\  almost forms a standard in the Scientific Community, especially due to its effective and straightforward mathematical capabilities.\\
    This is Chapter\ \ref{chap::intro}. If you want to know more about \LaTeX\ you better read
    \cite{texbook} or use the extensive help available on the internet. \index{LaTeX}. This 'hidden' index command helps you making an index at the end of your thesis. You can add this flag anywhere you want to make an index hit. You can see here also how to use acronyms, like \ac{DUT}. The acronyms are automatically listed in the corresponding section. Also, hyperlinks are created automatically with the developed class file, such that your digital PDF version of your thesis can be read dynamically.
 Have fun with \LaTeX\ and your M.Sc. research project and good luck! \\ \\
 
The purpose of the introduction is to tell readers why they should want to read what follows the introduction. This chapter should provide sufficient background information to allow readers to understand the context and significance of the problem. This does not mean, however, that authors should use the introduction to rederive established results or to indulge in other needless repetition. The introduction should (1) present the nature and scope of the problem; (2) review the pertinent literature, within reason; (3) state the objectives; (4) describe the method of investigation; and (5) describe the principal results of the investigation.