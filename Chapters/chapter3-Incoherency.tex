\chapter{Incoherency} \label{chap::incoherency}

When multiple sources are applied in one experiment the wavefields of the individual sources overlap. \cite{Mahdad-Deblending-Method} presented a method to separate the overlapping wavefields: A key step is pseudo-deblending, which yields the the desired unblended data superimposed by so called blending noise. This noise is generated by the overlap of the individual sources and can be removed with noise filters. However, if the individual wavefields overlap coherently, the blending noise will also be coherent and cannot be removed. In other words it is crucial to fire the sources incoherently.

In this chapter the incoherency of the source overlap is analyzed by considering three questions: Which factors control the incoherency? How can the incoherency be measured? How can the incoherency be maximized for an optimal deblending result?

\section{Incoherency Control Factors}

The deblended data can be represented by

\begin{equation}
	\mathbf{ P_{debl} } = \mathbf{X S \Gamma \Gamma ^H}.
	\label{eq:Ch-Incoherency-Deblended-Data}
\end{equation} 

Each of the matrices in the above equation influences the degree of incoherency of the source overlap. First, the contribution of the Earth \textbf{X} is neglected because it cannot be controlled in a seismic experiment. Second, the source signature \textbf{S}, in particular its time duration, determines the required minimum time delay between sources to avoid an overlap. Consequently, when designing an acquisition with simultaneous sources one must take into account the source signature. Thirdly, the incoherency is strongly dependent on the blending matrix $\mathbf{\Gamma}$ because it captures the firing pattern, i.e. it knows which sources are superimposed and the time delays between the sources in a given experiment. Therefore, the main focus of the incoherency analysis will be on the blending matrix $\mathbf{\Gamma}$.

\section{Quantification of Incoherency}

A mathematical tool to express incoherency is the autocorrelation function $R(\tau)$,

\begin{equation}
	R(\tau) = \int\limits_{\mathbb{R}} f^*(x) f(x + \tau) \, \text{d} x.
	\label{eq:Ch-Incoherency-Autocorrelation}
\end{equation}

For example, in a 1D case the autocorrelation of a fully incoherent function is a spike at zero lag. If a coherent or repetitive pattern is present in a function the autocorrelation will also have non zero amplitudes at other lags. 

For comparative purposes the incoherency of a function $f(x)$ should be quantified. It is suggested to measure incoherency $\mu$ as the ratio of the squared zero lag autocorrelation and the sum of the squared autocorrelation amplitudes,

\begin{equation}
	\mu = \frac{R(\tau = 0)^2}{\sum\limits_{\tau} R(\tau)^2}	.
	\label{eq:Ch-Incoherency-Incoherency}
\end{equation} 

This expression quantifies incoherency as a number between 0 and 1. A fully incoherent function $f(x)$ yields an autocorrelation which is a perfect spike. Thus, the ratio in equation \ref{eq:Ch-Incoherency-Incoherency} equals 1. For a perfectly coherent function $f(x)$ the ratio in equation \ref{eq:Ch-Incoherency-Incoherency} is nearly zero.

The incoherency strongly depends on the blending matrix $\mathbf{\Gamma}$ which is a 3D array. Before applying an autocorrelation the blending matrix $\mathbf{\Gamma}$ is transformed to time domain. The time domain blending matrix is denoted as $\mathbf{\gamma}$. It has the dimensions

\begin{equation}
	\text{dim}(\mathbf{\gamma}) = \text{Sources x Experiments x Time}.
	\label{eq:Ch-Incoherency-gamma}
\end{equation}

If a source $s_i$ is fired in an experiment $e_i$ at a time $t_i$ the element $(s_i,e_i,t_i)$ of $\mathbf{\gamma}$ is 1, else it is 0. If the source is not a perfect spike its amplitude can smear out across several time samples in the matrix $\mathbf{\gamma}$.

The incoherency of the blending matrix $\mathbf{\gamma}$ can be quantified by replacing the 1D autocorrelation in equation \ref{eq:Ch-Incoherency-Autocorrelation} and \ref{eq:Ch-Incoherency-Incoherency} with a 3D autocorrelation, which can be written as,

\begin{equation}
	R(\tau_1,\tau_2,\tau_3) = \iiint\limits_{\mathbb{R}^3} f^*(x,y,z) f(x + \tau_1,y + \tau_2, z + \tau_3) \, \text{d} x \, \text{d} y \, \text{d} z.
	\label{eq:Ch-Incoherency-3dAutocorrelation}
\end{equation}

\todo[inline]{The calculation of the 3D autocorrelation function requires significant computational power such that symmetry properties of the autocorrelation should be exploited to reduce the cost. Symmetry with respect to the origin, but using convn in Matlab I cannot access it}

The elements of the resulting 3D autocorrelation array are a measure for the correlation between wavefields at a specific source, experiment and time lag. For example, if in each experiment adjacent sources are fired with a constant time delay $\Delta t$, the 3D autocorrelation will yield a high amplitude at the source lag 1, experiment lag 0 and time lag $\Delta t$.

Combining equations \ref{eq:Ch-Incoherency-Incoherency} and \ref{eq:Ch-Incoherency-3dAutocorrelation} the incoherency of the time domain blending matrix $\mathbf{\gamma}$ can be expressed as,

\begin{equation}
	\mu = \frac{R(\tau_1 = 0,\tau_2 = 0,\tau_3 = 0)^2}{\sum\limits_{(\tau_1,\tau_2,\tau_3)} R(\tau_1,\tau_2,\tau_3)^2}	.
	\label{eq:Ch-Incoherency-Incoherency3d}
\end{equation} 

\section{Optimization of Incoherency}

\todo[inline]{Relate the incoherency estimate to the quality factor of the deblending. Point out that the quality factor is very sensitive to the time incoherency while the incoherency with respect to experiments or sources has less impact on the deblending performance.}

\section{Fingerprint of the Incoherency in the Blending Matrix $\mathbf{\Gamma}$}

To achieve a better understanding of the blending process the relation between the incoherency quantification and the blending matrix $\Gamma$ is assessed.

In frequency domain the blending matrix has the dimension,

\begin{equation}
	\text{dim}(\mathbf{\Gamma}) = \text{Sources x Experiments x Frequency},
	\label{eq:Ch-Incoherency-Dim-Gamma}
\end{equation}

where each element is a complex number $a \mathrm{e}^{ - \mathrm{j} \omega t }$.






















