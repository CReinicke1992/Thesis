\chapter{Results}

\section{Blending pattern}


\begin{itemize}
	\item Explain how the acquisition limits the firing pattern
	\item Introduce spatial incoherency
	\item Suggest the applied blending pattern
	\item Show quality factor versus incoherency/firing pattern
\end{itemize}

An incoherent blending pattern is important for good deblending performance (see section \ref{sec:BlendingMatrix}). Different incoherent blending patterns will be tested, in particular, temporally and spatially inherent blending patterns. 

As this thesis focuses on blended crossline sources (see Figure \ref{fig:Ch-Theory-3D-BlendedAcquisition}) the blending pattern already underlies some constraints. For example, the sources within one crossline must be blended before the vessel reaches the next inline position in order to achieve a sufficiently small inline source sampling rate. 

Based on these limits there are three possibilities to blend the sources incoherently. First, the sources can be blended with random time delays. Second, one can randomly pick sources from the crossline and blend them, which will be called spatial incoherency. Third, temporal and spatial incoherency can be combined, i.e. randomly picked sources are blended with random time delays.


\section{3D FKK Filter Performance}

\section{Feasibility}








